\documentclass[french]{article}
\usepackage[T1]{fontenc}
\usepackage{babel}
\usepackage[useregional=text]{datetime2}
\usepackage{geometry}
\usepackage{Style}
\usepackage{Commands}
\usepackage{fancyhdr}
\usepackage{titling}
\usepackage{lipsum}
\pagestyle{fancy}
\renewcommand{\headrulewidth}{0pt}

\setlength{\droptitle}{-2cm}

\geometry{
    a4paper,
    left=15mm,
    right=15mm,
    bottom = 25mm,
    top=20mm
}

\fancyhf{}
\lhead{\thepage}
%\chead{\thetitle}
\chead{\makebox[\textwidth][c]{\MakeUppercase{\thetitle}}}
\rhead{\makebox[\textwidth][r]{\tiny\thedate{}}}
%\rhead{\thedate}



\title{Title}
\date{\today{}}
\author{Author}
\university{University}

\begin{document}
\maketitle
    \section{Geometry}
    \lipsum[0-3]

    \section{Boxes}

    \subsection{Théorème}
    
    \begin{verbatim}
    \begin{theoreme}{Title}{reference}
        Text
    \end{theoreme}
    \end{verbatim}

    \begin{theoreme}{Title}{reference}
        Text
    \end{theoreme}

    Can be referred with 
    \begin{verbatim} 
        \ref{thm:reference}
    \end{verbatim}

    \subsection{Propriété}
    \begin{verbatim}
        \begin{propriete}{Title}{reference}
            Text
        \end{propriete}
    \end{verbatim}

    \begin{propriete}{Title}{reference}
        Text
    \end{propriete}

    Can be referred with 
    \begin{verbatim} 
        \ref{prop:reference}
    \end{verbatim}

    \subsection{Warning}

    \begin{verbatim}
        \begin{warning}
            Text
        \end{warning}
    \end{verbatim}

    \begin{warning}
        Text
    \end{warning}

    \subsection{Preuve}

    \begin{verbatim}
        \begin{preuve}
            Text
        \end{preuve}
    \end{verbatim}

    \begin{preuve}
        Text
    \end{preuve}

    \subsection{Exemple}
    

    \begin{verbatim}
        \exemple{Title}{Text}{Solution}
    \end{verbatim}

    \exemple{Title}{Text}{Solution}

    \begin{warning}
        Solution can be empty.
    \end{warning}
    
    \begin{verbatim}
        \exemple{Title}{Text}{}
    \end{verbatim}

    \exemple{Title}{Text}{}

\section{Commands}

\begin{tabular}{|c|c|c|}
    \hline
    \verb$\Real{Expr}$ & $\Real{Expr}$ & Real part of \verb|Expr| \\
    \verb$\Imag{Expr}$& $\Imag{Expr}$  & Imaginary part of \verb|Expr|\\
    \hline
    \verb$\Im{Expr}$ & $\Im{Expr}$ & Image of \verb|Expr| \\ 
    \verb$\Ker{Expr}$ & $\Ker{Expr}$ & Kernel of \verb|Expr| \\
    \hline
    \verb|$\R$| & $\R$ & Real \\
    \verb|$\C$| & $\C$ & Complex \\
    \verb|$\K$| & $\K$ & K \\
    \verb|$\N$| & $\N$ & Naturals \\
    \verb|$\Z$| & $\Z$ & Relatives \\
    \hline
    \verb$\PolyC$ & $\PolyC$ & Complex Polynom \\ 
    \verb$\PolyC[deg]$ & $\PolyC[2]$ & Complex Polynom (deg inferior to deg) \\
    \verb$\PolyR$ & $\PolyR$ & Real Polynom \\ 
    \verb$\PolyR[deg]$ & $\PolyR[2]$ & Real Polynom (deg inferior to deg) \\
    \verb$\Poly$ & $\Poly$ & $\K$ Polynom \\ 
    \verb$\Poly[deg]$ & $\Poly[2]$ & $\K$ Polynom (deg inferior to deg) \\
    \hline
    \verb$\MatC$ & $\MatC$ & Complex matrix \\ 
    \verb$\MatC[deg]$ & $\MatC[2]$ & Complex matrix (deg inferior to deg) \\
    \verb$\MatR$ & $\MatR$ & Real matrix \\ 
    \verb$\MatR[deg]$ & $\MatR[2]$ & Real matrix (deg inferior to deg) \\
    \verb$\Mat$ & $\Mat$ & $\K$ matrix \\ 
    \verb$\Mat[deg]$ & $\Mat[2]$ & $\K$ matrix (deg inferior to deg) \\
    \hline
    \verb$\Matbf{f}$ & $\MatBf{f}$ & \\
    \verb$\Matbf[\N]{f}$ & $\MatBf[\N]{f}$ & \\
    \hline
    \verb|$\Serie[var][start]{Expr}$| & $\Serie[var][start]{Expr}$ & (default: var=k, start=0)\\
    \verb|$\Int[start][var]{Expr}$| & $\Int[start][var]{Expr}$ & (default: start=0, var=x)\\
    \verb|$\Lim[var][to]{Expr}$| & $\Lim[var][to]{Expr}$ & (default: var=x, to=$+\infty$)\\
    \verb|$\diff[var][order]{Expr}$| & $\diff[var][order]{Expr}$ & (default: var=x, order=$\emptyset$)\\
    \verb|$\diff[var]{Expr}$| & $\diff[var]{Expr}$ & \\
    \hline
    \verb|$\lvec$| & $\lvec$ & \\
    \verb|$\cvec$| & $\cvec$ & \\
    \hline
    \verb|$\App[Space][Expr]$| & $\App[Space][Expr]$ & (default: Space=E, Expr=$\emptyset$) \\
    \hline
\end{tabular}

\section{Aliases}

\begin{tabular}{|c|c|}
    \hline
    \verb|\ds| & \verb|\displaystyle| \\
    \verb|\bb{}| & \verb|\mathbb{}| \\
    \verb|\cal{}| & \verb|\mathcal{}| \\
    \verb|\t{}| & \verb|\text{}| \\
    \hline
\end{tabular}

\end{document}